% Chapter 1

\chapter{Planteamiento del problema} % Main chapter title
\label{PlanteamientoProblema} % For referencing the chapter elsewhere, use \ref{Chapter1} 

\lhead{Capítulo \ref{PlanteamientoProblema}. \emph{Planteamiento del problema}} % This is for the header on each page - perhaps a shortened title

Un sistema musculo-esquelético dinámico consiste de varias estructuras biológicas (huesos, músculos, tendones, ligamentos, tejido conectivo, etc.), y es un componente esencial para la simulación y análisis del movimiento de los humanos. Sin embargo, simular esos sistemas es algo complicado debido a la inherente complejidad tanto del sistema como de los elementos que lo componen. Los simuladores existentes típicamente modelan a los músculos como objetos rígidos, uniendo las propiedades físicas de estos con otras estructuras cercanas, como los huesos, y se les aplican modelos fenomenológicos para simular su activación. Aunque hay varios trabajos que aplican modelos biofísicos, que intentan simular la estructura correcta de los músculos y su interacción con las otras estructuras del sistema musculo-esquelético, se enfocan en partes específicas del sistema, como la simulación de las fibras musculares.

Hay muchos retos que se tienen que resolver antes de poder simular adecuadamente el sistema musculo-esquelético. Algunos de los principales se mencionan a continuación:

\begin{itemize}
	\item La mayoría de las simulaciones existentes simplifican el sistema musculo-esquelético, ya sea simplificando su arquitectura, sus componentes internos, o su forma de activación y control.
	\item Casi todos los tejidos del sistema son deformables, y tienen propiedades físicas. Algunos efectos como el abultamiento de los músculos activos depende de el hecho de que los músculos son sólidos incompresibles. En la mayoría del trabajo previo, el tejido volumétrico se ignora o se simula utilizando técnicas como FEM.
	\item Simuladores basados en mecánicas de sólidos, como los de FEM, no son ideales debido a que requieren una detección de colisiones y de resolución de conflictos, que son computacionalmente caras y no son ideales para objetos deformables. Además, las primitivas básicas, tetraedros y hexaedros, no son similares a los elementos básicos de la deformación muscular: las fibras musculares.
	\item Los músculos están en contacto muy cercano entre ellos. Procesar el contacto interno entre dos mallas en tres dimensiones puede ser caro computacionalmente. Por esto, muchos trabajos evitan la simulación de contacto interno entre músculos, y externo con elementos de un escenario. 
	\item Los principales métodos para controlar la activación de los músculos son fenomenológicos. Estos son simplificaciones del comportamiento real de los músculos, y se ha demostrado que tienen un nivel de error fuerte al compararlo con datos de activación de músculos reales. Esos modelos son más parecidos a sistemas mecánicos simples, y solo modelan la fuerza generada entre dos puntos, el origen y la inserción del músculo, es decir, modelan a un objeto de una dimensión. Además, los músculos son controlados por el sistema nervioso, y hay pocos trabajos que se centran en simular su activación de esa forma.
	\item Aunque hay mucho avance en la simulación de las fibras musculares, estos trabajos no consideran las propiedades físicas de las fibras (entre las más importantes, la masa), y aunque hay varias técnicas para obtener su correcta arquitectura dentro del músculo, tienen problemas para obtener los puntos correctos de unión, o no representan correctamente su contribución a la forma del músculo. Además, es complicado modelar distribuciones complejas de fibras musculares.
	\item Los huesos se conectan a los músculos mediante tendones, y a otros huesos mediante ligamentos. Los modelos actuales simplifican dichas estructuras, o no las consideran. Sin embargo, es importante simular la dinámica del sistema acoplando dichas estructuras para poder mover a los personajes, y poder modelar efectos como el impacto de elementos adicionales sobre ellos.
\end{itemize}

Actualmente no existe un modelo del sistema musculo-esquelético que simule a los músculos del cuerpo humano considerando su estructura interna, los tejidos biológicos que los componen y sus propiedades físicas, y que se activen utilizando modelos biofísicos similares a la activación mediante el sistema nervioso. Además, los trabajos previos se enfocan en modelar un músculo, o un grupo de músculos, sin buscar generar métodos más generales que permitan modelar una gran variedad de músculos.  Un ejemplo de esto son los trabajos realizados utilizando FEM, ya que tienen una limitación fuerte en cuanto a la forma de los músculos, ya que los más complejos no son fáciles de modelar utilizando hexaedros o tetraedros. De igual forma, varios trabajos simplifican los músculos al grado que sólo se modelan como un segmento de línea, sin importar su forma o configuración en tres dimensiones.

Adicionalmente, los costos computacionales para resolver los diferentes modelos matemáticos, así como para graficar los músculos de la manera más realista posible, hacen que obtener simulaciones en tiempo real sea difícil; muy pocos trabajos utilizan GPGPU para paralelizar los cálculos relacionados con las simulaciones con el fin de poder ejecutarlas en tiempo real. Esto hace que sea complicado utilizar los trabajos propuestos en soluciones más interactivas, como simuladores de entrenamiento o de aprendizaje de anatomía y medicina, o en videojuegos o películas animadas con personajes humanoides. 

%----------------------------------------------------------------------------------------
%El modelado matemático de músculos es un área poco estudiada debido a la complejidad inherente. Aunque hay datos de varios músculos, no existen conjuntos de datos que provean parámetros completos de los músculos de todo el cuerpo. Otro punto importante es que, en la mayoría de los trabajos ya realizados, se utiliza el modelo de músculos de Hill, que es una simplificación del funcionamiento real de los músculos, utilizando solo un segmento de línea por músculo. 
%
%Los cálculos de cinemática directa e inversa usando transformaciones homogéneas son comúnmente utilizados para el modelado matemático de los movimientos de las articulaciones y eslabones del cuerpo humano. De igual manera, los modelos biomecánicos de los músculos, aunque dan un resultado similar al comportamiento de los músculos, no se basan en la forma y funcionamiento real de estos, si no en un modelo mecánico (sistema masa-resorte-amortiguador) que tiene un comportamiento similar.
%
%Sin embargo, utilizando solamente esos modelos matemáticos no se pueden obtener movimientos simulados reales similares a los que puede realizar el cuerpo humano. Debido a la complejidad que implica un modelado correcto, considerando la anatomía y biomecánica del cuerpo, los investigadores se han alejado del desarrollo de un modelado biomecánico del cuerpo completo.
%
%El problema es que, actualmente, no existe un modelo matemático de los músculos del cuerpo humano, basado en su forma, estructura, y función anatómica. Los pocos modelos existentes utilizan estructuras mecánicas para intentar aproximarse al comportamiento de los músculos. Igualmente, no hay expertos que sean capaces de explicar como funcionan matemáticamente los músculos sin utilizar modelos mecánicos como base.
%
%Al no existir un modelo que sea capaz de simular los músculos de manera real, tampoco existen simulaciones ni animaciones que sean capaces de reproducir movimientos y formas de manera natural.
