\section{Muscle Visualization Module}
In order to test, evaluate and later validate the proposed method, we started the development of a visualization software tool that is able to display geometry efficiently by using parallel hardware (GPU) techniques; the visualization software will also provide an interface to control simulation parameters.

%----------------------------------------------------------------------------------------

\subsection{Visualization Implementation}

By using a set of third party software tools, we assembled a rendering engine (Figure~\ref{fig:muscleVis}) that allows efficient display and control of a simulation that uses the proposed method. Open Source libraries for rendering, shading, graphical user interface, texture and geometry loading have been integrated to the visualization software; these tools are listed in Table~\ref{tab:thirdSw}.

\begin{figure}[!t]
	\centering
		\includegraphics[width=0.8\textwidth]{./Figures/viewConfig.jpg}
	\caption[Muscle rendering.]{Muscle rendering tool.}
	\label{fig:muscleVis}
\end{figure}

\begin{table}[htbp]
  \centering
  \caption{Third party software for virtual muscle rendering}
    \begin{tabular}{rr}
    \toprule
    \multicolumn{2}{c}{\textbf{Third party software tools}} \\
    \midrule
    \textbf{Rendering Library} & OpenGL \\
    \textbf{Shading} & GLSL \\
    \textbf{GUI} & Qt 5.4.1 Community \\
    \textbf{Texture loader} & DevIL \\
    \textbf{Geometry loader} & ASSIMP \\
    \bottomrule
    \end{tabular}%
  \label{tab:thirdSw}%
\end{table}%

\subsection{Rendering Advances}

From the BodyParts3d~\citep{bodyParts3d} database, we have identified, colored and separated virtual muscle geometries (Figure~\ref{fig:muscleView}) of interest\textemdash right biceps brachii, brachialis, brachiordialis, pronator teres, triceps brachii, anconeus\textemdash that will serve as input for the muscle simulation. Together, all the arm meshes are composed of 271,004 faces which are rendered at interactive frame rates: up to 0.00588235 seconds per frame using a Nvidia GeForce 710M GPU as shown in Figure~\ref{fig:muscleRendering}, thus allowing for simulation computations to be performed even on the same hardware.

\begin{figure}[t]
    \centering
    \begin{subfigure}[t]{0.45\textwidth}
        \centering
        \includegraphics[width=\textwidth]{./Figures/musclesFront.jpg}
        \caption{Blue: biceps brachii. Purple: brachialis. Green: brachiordialis. Aqua: pronator teres.}
        \label{fig:musclesFront}
    \end{subfigure}
\hfill
    \begin{subfigure}[t]{0.45\textwidth}
        \centering
        \includegraphics[width=\textwidth]{./Figures/musclesBack.jpg}
        \caption{Yellow: triceps brachii. Gray: anconeus.}
        \label{fig:musclesBack}
    \end{subfigure}

    \caption{Muscle geometry}
    \label{fig:muscleView}
\end{figure}

\afterpage{
\begin{figure}[t]
    \centering
    \begin{subfigure}[t]{0.45\textwidth}
        \centering
        \includegraphics[width=\textwidth]{./Figures/visFPS.jpg}
        \caption{Interactive rate rendering.}
        \label{fig:visFPS}
    \end{subfigure}
	\hfill
    \begin{subfigure}[t]{0.45\textwidth}
        \centering
        \includegraphics[width=\textwidth]{./Figures/transparencyConfig.jpg}
        \caption{Transparency control.}
        \label{fig:transparency}
    \end{subfigure}

    \caption{Muscle rendering}
    \label{fig:muscleRendering}
\end{figure}
}
%\begin{figure}
%    \centering
%    \begin{subfigure}[t]{0.45\textwidth}
%        \centering
%        \includegraphics[width=\textwidth]{./Figures/musclesFront.jpg}
%        \caption{Blue: biceps brachii. Purple: brachialis. Green: brachiordialis. Aqua: pronator teres.}
%        \label{fig:musclesFront}
%    \end{subfigure}
%\hfill
%    \begin{subfigure}[t]{0.45\textwidth}
%        \centering
%        \includegraphics[width=\textwidth]{./Figures/musclesBack.jpg}
%        \caption{Yellow: triceps brachii. Gray: anconeus.}
%        \label{fig:musclesBack}
%    \end{subfigure}
%
%    \caption{Muscle geometry}
%    \label{fig:muscleView}
%\end{figure}
%
%
%\begin{figure}
%    \centering
%    \begin{subfigure}[t]{0.45\textwidth}
%        \centering
%        \includegraphics[width=\textwidth]{./Figures/visFPS.jpg}
%        \caption{Interactive rate rendering.}
%        \label{fig:visFPS}
%    \end{subfigure}
%	\hfill
%    \begin{subfigure}[t]{0.45\textwidth}
%        \centering
%        \includegraphics[width=\textwidth]{./Figures/transparencyConfig.jpg}
%        \caption{Transparency control.}
%        \label{fig:transparency}
%    \end{subfigure}
%
%    \caption{Muscle rendering}
%    \label{fig:muscleRendering}
%\end{figure}