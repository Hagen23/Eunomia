\section{Hipótesis} % Main chapter title

\label{HipotesisObjetivos} % For referencing the chapter elsewhere, use \ref{Chapter1} 

%\lhead{Capítulo \ref{HipotesisObjetivos}. \emph{Hipótesis}} % This is for the header on each page - perhaps a shortened title

Se busca demostrar que el desarrollar un modelo del sistema musculo-esquelético, basándose en su arquitectura anatómica considerando los huesos, fibras musculares, tejidos conectivos, tendones, entre otros tejidos, y usando un modelo de activación y control biofísico, podría permitir la generación de simulaciones gráficas dinámicas y en tiempo real de miembros del cuerpo humano cuyo comportamiento se asemeje al comportamiento real de su contra parte en un ser humano.

%----------------------------------------------------------------------------------------
%Se busca demostrar que el realizar un modelo matemático y biomecánico de los músculos esqueléticos de los miembros superiores del cuerpo, interactuando sobre el esqueleto correspondiente, basándose en su forma, estructura y función, podría generar simulaciones gráficas de  manera dinámica y en tiempo real que se asemejen al comportamiento real de estos en el cuerpo humano.

\section{Objetivo General}

Desarrollar un modelo del sistema musculo-esquelético basado en su arquitectura anatómica y en modelos biofísicos. Además, crear un marco de trabajo para generar simulaciones de miembros del cuerpo humano, basadas en el modelo previo, que interactúen con un ambiente virtual.

%Realizar un modelo biofísico detallado de los músculos esqueléticos de los miembros superiores del cuerpo, interactuando sobre el esqueleto correspondiente, así como crear un marco de trabajo que permita generar simulaciones y animaciones gráficas de personajes humanoides interactuando con un ambiente virtual. 

\section{Objetivos específicos}

\begin{itemize}
	\item Generar el modelo en base a datos reales de músculos esqueléticos.
	\item Modelar las diferentes estructuras del sistema músculo-esquelético (fibras musculares, tejidos conectivos, tendones, y piel) como sólidos Euleriano-Lagrangianos.
	\item Simular un miembro superior con los músculos Biceps Brachii, Brachiordialis, Brachialis, Pronator Teres, Triceps Brachii, y Anconeus, con el fin de que realice los movimientos de flexión y extensión en el codo.
	\item Programar los algoritmos necesarios para el modelo en paralelo con CUDA para aprovechar el GPU. Esto debido a la cantidad de cálculos que se necesitarán realizar y a que se requiere que las simulaciones se ejecuten en tiempo real.
	%\item Aplicar el modelo obtenido para modelar y simular diferentes músculos esqueléticos del cuerpo.
	\item Validar las simulaciones realizadas utilizando una prueba de Turing, y comparándolas con datos de electromiografías y de captura de movimiento de movimientos similares a los realizados en las simulaciones.
\end{itemize}

\section{Motivación}

De las principales áreas de aplicación está la medicina, donde un modelo musculo-esquelético correcto, capaz de emular precisamente el comportamiento real de los  músculos del cuerpo humano, puede ayudar a probar terapias nuevas, puede ayudar a generar simuladores de operaciones para entrenamiento, así como para ayudar en la generación de diagnósticos más acertados, entre otras muchas aplicaciones. 

Aplicar un modelo musculo-esquelético a humanos virtuales, cuyo comportamiento sea más similar al que se presenta en un ser humano, también podría ayudar en simulaciones de choques para probar automóviles, se podría utilizar para el mundo de la moda, de los juegos de video y películas, para generar programas de enseñanza a distancia y presentar profesores simulados más reales, se puede usar para generación de robots humanoides con un comportamiento más real, así como para simulaciones que involucren el uso de robots, entre otras aplicaciones posibles.

El diseñar un modelo que se pueda utilizar en distintas áreas y proyectos, puede ayudar en varios sentidos: permitiría reducir los costos relacionados con el desarrollo de productos y servicios así como mejorar su eficiencia y precisión, podría permitir la mejora de la atención médica, ayudaría a obtener una mayor calidad en animaciones, juegos de video, películas, etc.

Ahí es donde se justifica la importancia del trabajo: el trabajo propuesto no se confina a un área del conocimiento, sino que se puede aplicar en distintos lugares para generar simulaciones reales y resultados certeros.

